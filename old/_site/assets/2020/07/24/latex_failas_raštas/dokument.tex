\documentclass[12pt]{letter}

\usepackage[a4paper,nomarginpar,total={170mm,257mm},left=25mm, right=25mm, top=20mm, bottom=20mm,]{geometry}

\usepackage{setspace}


\usepackage[L7x]{fontenc}
\usepackage[lithuanian]{babel}
\usepackage[utf8]{inputenc}

\usepackage{textcomp}% Supports many additional symbols
\usepackage{amsmath}% Math equations, etc.
\usepackage{amsfonts}% Math fonts (e.g. script math fonts)
\usepackage{amssymb}% Math symbols such as infinity
\usepackage{amsthm}
\usepackage{lmodern}

\usepackage{enumitem}% http://ctan.org/pkg/enumerate
\usepackage{wrapfig}
\usepackage{graphicx}
\usepackage{subcaption}
\usepackage{float}
\usepackage{hyperref}
\hypersetup{
  colorlinks=true,
  linkcolor=black,
  filecolor=blue,
  urlcolor=blue,
  citecolor=blue
}
%\usepackage{draftwatermark}
%\SetWatermarkText{Pirminė versija}
%\SetWatermarkScale{0.5}

\begin{document}

\begin{flushright}
Vilnius, 2020-07-24
\end{flushright}



Lietuvos Respublikos Ministrui Pirmininkui Sauliui Skverneliui\\
Lietuvos Respublikos finansų ministerijai\\
\\
Kopija:\\
\\
Lietuvos Respublikos Prezidentui J.E. Gitanui Nausėdai\\        
Lietuvos Respublikos Seimo Pirmininkui Viktorui Pranckiečiui\\ 
Lietuvos Respublikos Seimo nariams\\
Centrinei projektų valdymo agentūrai\\
Žiniasklaidos atstovams\\

\textbf{Profesinių sąjungų, nevyriausybinių organizacijų bei visuomenininkų kreipimasis dėl ateities ekonomikos DNR plano  Lietuvai veiksmų ir projektų įgyvendinimo tvarkos bei galimos korupcijos rizikos ją įgyvendinant}

Gerbiamas Premjere Sauliau Skverneli,

Kreipiamės į Jus, kaip į Lietuvos Respublikos Vyriausybės vadovą, atsakingą už visos Vyriausybės ir jai pavaldžių institucijų veiklą. Šiuo raštu norime atkreipti Jūsų, visų politikų bei visuomenės dėmesį į tai, jog, mūsų vertinimu, DNR plano Lietuvai įgyvendinimo procesas neatitinka net minimalių skaidrumo ir teisėtumo reikalavimų.

Daugiausiai abejonių kelia tai, jog projektai pradedami, tvirtinami ir finansavimas jiems skiriamas galimai neskaidriai, nėra teikiamas atnaujintas 2020m. Valstybės biudžetas, nevyksta jokia valstybės pajamų ir išlaidų parlamentinė kontrolė, o į DNR plano rengimą ir įgyvendinimą nėra įtraukiami nepriklausomi ekspertai, dalis svarbių nevyriausybinių organizacijų bei viešojo sektoriaus darbuotojus atstovaujančios profesinės sąjungos.

Siekiant išsklaidyti visus neaiškumus, žemiau pateikiame aktualiausius klausimus ir prašome į juos kaip įmanoma skubiau ir išsamiau atsakyti. Suprantame, jog Vyriausybė susiduria su neplanuota COVID-19 infekcijos krize, tačiau krizės faktas negali būti naudojamas neskaidriai veiklai pridengti.



\textbf{KLAUSIMAI:}


\renewcommand{\labelenumi}{\Roman{enumi}}
\renewcommand{\labelenumii}{\arabic{enumii}}
\renewcommand{\labelenumiii}{\alph{enumiii}}


\begin{enumerate}
\item \textbf{Proveržio grupės}

\begin{enumerate}
\item Prašome pateikti visų „proveržio grupių“ dalyvių sąrašą, nurodant asmenį, dalyvavusį proveržio grupėje, instituciją / įmonę, kuriai atstovavo proveržio grupėje dalyvavęs asmuo, sąrašą projektų / koncepcijų, kurias pateikė konkretus asmuo / institucijos atstovas bei ministeriją, kurios proveržio grupėje minimas asmuo dalyvavo.

\item Prašome  pateikti sąrašą, kuriame nurodyta, kokius konkrečius narius DNR plano organizatorius siūlė ministerijoms įtraukti į šias grupes ir kokiais kriterijais remiantis buvo teikti pasiūlymai.

\item Prašome  pateikti sąrašą institucijų, pakviestų prisijungti (nebūtinai faktiškai prisijungusių) prie „proveržių grupių“, nepriklausomai nuo to, ar kvietimas buvo išsiųstas DNR plano koordinatoriaus, ar proveržio grupę organizuojančios konkrečios ministerijos.
\end{enumerate}


\newpage
\item \textbf{Projektų plėtotojai}
\begin{enumerate}[resume]
\item Prašome  pateikti  sąrašą projektų plėtotojų, kurie rengė veiksmo įgyvendinimo planus bei konkrečius kiekvieno projekto plėtotojo pateiktus veiksmo planus. Jeigu projekto plėtotojas buvo juridinis asmuo, pvz., asociacija, tada kartu prašome pateikti tokios institucijos atstovaujamų juridinių asmenų / narių sąrašą. Jeigu projekto plėtotojas buvo fizinis asmuo, prašome nurodyti fizinio asmens darbovietę.
\end{enumerate}

\item \textbf{Investicijų komiteto nariai}
\begin{enumerate}[resume]

\item Prašome  pateikti liepos 10 d. tarpinstitucinio ministerijų posėdžio video / garso įrašą ir stenogramą bei protokolą, kuriame buvo nuspręsta, kokius socialinius bei ekonominius partnerius ministerijos siūlo pakviesti dalyvauti Investicijų komiteto veikloje.

\item Prašome paaiškinti, dėl ko susiklostė tokia situacija, jog Investicijų komiteto / proveržio grupėse nedalyvauja nei skėtinės, nei atitinkamų sferų, pvz., švietimo, profesinės sąjungos?

\item Prašome  pateikti Investicijų komiteto dalyvaujančių asmenų, jų atstovaujamų institucijų bei šių institucijų (pvz., asociacijų) atstovaujamų juridinių asmenų sąrašą.
\end{enumerate}



\item \textbf{Investicijų komiteto veikla}

\begin{enumerate}[resume]
\item Pirmojo Investicijų komiteto posėdžio metu, pristatant komiteto veiklą, buvo pasakyta, jog komitetas balsuos ir priims sprendimus daugumos balsų persvara. Prašome patikinti, jog socialiniai ir ekonominiai partneriai negalės balsuoti, laikantis Valstybės nutarimo Nr. 750 5.3 nuostata, pagal kurią šie partneriai turi tik patariamąją teisę komiteto veikloje.

\item Prašome paaiškinti, ar kiekviena valstybinė institucija turės vieną balsą balsuojant Investicijų komiteto pakomitetyje, atsižvelgiant į tai, jog dauguma ministerijų yra delegavusios skirtingą asmenų skaičių į Investicijų komitetą? 

\item Prašome patikinti, jog visų ministerijų atstovai bus reprezentuojami visuose Investicijų komiteto pakomitečiuose, priimant sprendimus dėl vykdomų koncepcijų. 

\item Prašome paaiškinti, kaip yra arba bus užtikrinama, jog „proveržio grupių“ nariai ir projekto plėtotojai neveiktų Investicijų komitete, dėl ko gali kilti korupcijos rizika: pirma teikiant koncepcijas / projektus ministerijoms ir vėliau pačių teiktas koncepcijas / projektus advokataujant (patarinėjant sprendimų priėmėjams) Investicijų komitete?

\item Siekiant kuo didesnio skaidrumo šiame projekte, prašome užtikrinti, jog viešai būtų prieinami visi Investicijų komiteto bei jo pakomitečių posėdžių vaizdo arba garso įrašai, galimai net sudarant galimybę visus posėdžius tiesiogiai stebėti per Vyriausybės Youtube.com kanalą. Taip pat prašome, kad visa su Investicijų komiteto veikla susijusi techninė projektų / koncepcijų dokumentacija (įskaitant ir kaštų naudos analizes) būtų pateikta viešai LR Vyriausybės tinklapyje, kaip ir detalūs Investicijų komiteto narių balsavimų dėl projektų rezultatai.
\end{enumerate}

\newpage 
\item \textbf{Jau vykdomi projektai}
\begin{enumerate}[resume]
\item Prašome pateikti visų jau vykdomų DNR plano Lietuvai projektų sąrašą (už 4.1 mlrd eurų + 699 mln eurų), kartu pateikiant kiekvieno projekto kaštų ir naudos analizę. Taip pat kartu nurodant kiekvieno projekto įgyvendinimo pradžią, numatoma pabaigą, projekto finansavimo apimtį, finansavimo šaltinius bei projekto vykdytoją.

\item Prašome patikslinti, ar nebuvo nusižengta įstatymams, nepateikiant atnaujinto biudžeto, kai jau vykdomiems projektams buvo skirtas papildomas 699 milijonai eurų finansavimas? Taip pat prašome pateikti informaciją, kokiais kriterijais buvo remtasi atrenkant projektus, kuriems skirtas papildomas finansavimas.

\end{enumerate}
\end{enumerate}



Išsamių atsakymų į šį raštą su visais duomenimis laukiame įstatymo numatyta tvarka ir prašome visų institucijų patvirtinti šio rašto gavimą, nurodant dokumentui suteiktą registracijos numerį bei registracijos datą. 

Sąrašus prašome pateikti tabeliniuose dokumentuose MS Excel (.xls, .xslx) arba atviro kodo tabelinių dokumentų (.ods) formatu.

Prašome Lietuvos Respublikos Vyriausybę bei Finansų ministeriją koordinuoti atsakymą į šį raštą vieno langelio principu, atsakymą persiunčiant visiems rašto pabaigoje nurodytiems pasirašiusiems asmenims. 

Taip pat norime atkreipti dėmesį, jog nei viena institucija / asmenys nėra gavę atsakymo iš Finansų ministerijos ir Lietuvos Respublikos Vyriausybės į 2020-05-25 išsiųstą  raštą dėl DNR plano Lietuvai.
\\
\\

\textbf{Kreipimąsi pasirašo:}
\spacing{1.5}

Gegužės 1-osios profesinės sąjungos atstovas Kostas Jakeliūnas
\includegraphics[width=0.1\textwidth]{kostas_jakeliunas.png}

Jaunųjų gydytojų asociacijos prezidentė Kristina Norvainytė                        
\includegraphics[width=0.1\textwidth]{kristina_norvainyte.png}

Lietuvos negalios organizacijų forumo prezidentė Dovilė Juodkaitė
\includegraphics[width=0.075\textwidth]{dovile_juodkaite.png}

Lietuvos profesinės sąjungos “Solidarumas” pirmininkė Kristina Krupavičienė
\includegraphics[width=0.1\textwidth]{kristina_krupaviciene.jpeg}

Lietuvos švietimo darbuotojų profesinės sąjungos pirmininkas Andrius Navickas
\includegraphics[width=0.15\textwidth]{andrius_navickas.png}

Lithuanian-Economy.net ekonomistas Justas Mundeikis
\includegraphics[width=0.1\textwidth]{justas_mundeikis.png}

Santaros klinikų slaugos darbuotojų profesinės sąjungos pirmininkas \\ \hspace*{1cm}Mingaudas Busevičius
\includegraphics[width=0.1\textwidth]{mingaudas_busevicius.jpg}

\spacing{2}
Studentų judėjimo „Šauksmas“ atstovas Domas Lavrukaitis
\includegraphics[width=0.1\textwidth]{domas_lavrukaitis.png}



\end{document}
